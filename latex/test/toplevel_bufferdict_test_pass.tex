\documentclass{article}

% Engine-specific settings
% pdftex:
\ifcsname pdfmatch\endcsname
    \usepackage[T1]{fontenc}
    \usepackage[utf8]{inputenc}
\fi
% xetex:
\ifcsname XeTeXinterchartoks\endcsname
    \usepackage{fontspec}
    \defaultfontfeatures{Ligatures=TeX}
\fi
% luatex:
\ifcsname directlua\endcsname
    \usepackage{fontspec}
\fi
% End engine-specific settings

\usepackage{latex2pydata}

\begin{document}

Buffer test.

% Alternative to \pydatasetfilename:
%\newwrite\testdata
%\immediate\openout\testdata=\jobname.pydata\relax
%\pydatasetfilehandle{\testdata}

\pydatasetfilename{\jobname.pydata}
\pydatasetbuffername{testbuffer}

\pydatawritedictopen
\pydatabufferkeyvalue{key1}{value1}
\pydatabufferkey{key2}
\pydatabuffervalue{value2}
\pydatabufferkey{key3}
\begin{pydatabuffermlvalue}
multiline
value
"double quote"
backslashes \ \ \\
\end{pydatabuffermlvalue}
\pydatawritebuffer
\pydatawritekey{buffermd5}
\edef\hash{\pydatabuffermdfivesum}
\expandafter\pydatawritevalue\expandafter{\hash}
\pydatawritedictclose

\pydataclosefilename{\jobname.pydata}

\end{document}
